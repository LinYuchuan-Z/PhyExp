%!TeX program = xelatex
\documentclass[12pt,hyperref,a4paper,UTF8]{ctexart}
\usepackage{zjureport}

%%-------------------------------正文开始---------------------------%%
\begin{document}

%%-----------------------封面--------------------%%
\cover

%%------------------摘要-------------%%
%\begin{abstract}
%
%在此填写摘要内容
%
%\end{abstract}

\thispagestyle{empty} % 首页不显示页码

%%--------------------------目录页------------------------%%
\newpage
\begin{spacing}{2}
\tableofcontents
\end{spacing}
\thispagestyle{empty}
\paragraph{}
\newpage
%%------------------------正文页从这里开始-------------------%


%%可选择这里也放一个标题
\begin{center}
    \title{ \Huge \textbf{{不规则物体的转动惯量测定方法}}}
\end{center}
\par
\begin{center}
    \textbf{Pan,He,Zhou,Zhou}
\end{center}
\paragraph{}

\section{背景知识}
\subsection{理论背景}

转动惯量又称为惯性矩,是物体的固有属性,描述了其在转动过程中对角速度变化的抵抗能力。在力学中,转动惯量通常用公式$I=\int r^2\text dm$表示,其中$r$是质量元$\text dm$到旋转轴的距离。它不仅取决于物体的质量分布,还与旋转轴的位置密切相关,因此轴的位置对转动惯量测量具有重要意义。

转动惯量的测量通常利用角动量守恒定律和牛顿第二定律的旋转形式。根据角动量定理,当合外力矩作用在刚体上时,刚体的角加速度$\alpha$与外力矩&\tau&之间满足$\tau=I\alpha$。通过已知外力矩和角加速度,可以反推刚体的转动惯量。

转动惯量的测定并非容易的问题,尤其是针对形状不规则的物体或拼接体。因此,采用扭摆法、三线摆法或落体法等间接方法在转动惯量的测定中备受欢迎。

\subsubsection{扭摆法}

扭摆法是一种利用扭转振动测量物体转动惯量的经典方法,其理论原理基于单摆运动与刚体转动的结合。扭摆的基本结构由悬挂刚体的细长弹性丝或扭转弹簧组成,当刚体被轻微旋转后,系统会产生周期性的扭转振动。
\begin{enumerate}
    

\item 扭转振动的基本原理

当刚体绕扭丝的轴发生微小转动时,弹性扭丝会产生恢复力矩$\tau$来抵抗这种转动,恢复力矩与角位移$\theta$成正比,即:

$$ \tau = -k \theta $$

根据刚体转动的牛顿第二定律:

$$ \tau = I \alpha $$

$$ I \frac{d^2\theta}{dt^2} + k\theta = 0 $$

该方程描述了刚体绕固定轴的简谐振动,其振动周期$T$为:

$$T = 2\pi \sqrt{\frac{I}{k}}$$

\item 测量扭转刚度 

扭转刚度$k$是弹性丝的固有属性,与其材料、直径、长度等参数有关。在实验中,可以通过已知转动惯量的标准物体测量扭转刚度:

$$k = \frac{4\pi^2 I_{\text{标准}}}{T_{\text{标准}}^2}$$

\item 测量未知物体的转动惯量

将待测物体安装在扭丝上,测量其扭转振动的周期 。通过已知的$T$值,利用下式计算转动惯量:

$$I = \frac{k T^2}{4\pi^2}$$

\item 复合物体的转动惯量

如果测量系统由扭丝、标准物体和待测物体组成,总转动惯量为:

$$I_{\text{总}} = I_{\text{标准}} + I_{\text{待测}}$$

\item 误差分析

使用扭摆法测量转动惯量时,主要误差来源包括:

扭丝非线性恢复力矩导致的系统误差。

振动周期测量中的读数误差。

扭丝刚度$k$计算误差。

系统外界扰动对振动周期的影响。


通过多次测量取平均值、减小扭丝非线性效应及减小振幅等措施可以有效降低误差。

\end{enumerate}


\textbf{总结}

扭摆法通过测量振动周期并结合扭转刚度来精确计算物体的转动惯量。其优点在于结构简单、实验精度高,适用于多种刚体转动惯量的测量。

\subsubsection{三线摆法}

三线摆法是一种测量物体转动惯量的实验方法,其特点是结构简单、测量精度高,广泛应用于实验力学中。三线摆通过悬挂系统产生扭转振动,通过测量振动周期,结合系统的几何和物理参数计算待测物体的转动惯量。

1. 三线摆的基本结构

三线摆由三个等长且均匀分布的细线(或金属丝)悬挂一个刚体构成。细线的另一端固定在支架上。系统绕通过刚体重心且与三线平面垂直的轴进行自由扭转振动。刚体的重心与三线摆系统的对称性和力学平衡性保证了振动的稳定性。

2. 振动原理

三线摆的运动是绕垂直于线摆平面轴的扭转振动,其振动由刚体的转动惯量$I$与细线的弹性扭转力共同决定。当刚体偏离平衡位置发生转动时,细线的扭转恢复力矩使其产生简谐振动。

恢复力矩

细线因扭转产生的恢复力矩$\tau$与角位移$\theta$成正比:

$$ \tau = -k_{\text{总}} \theta $$

$$ k_{\text{总}} = 3k $$

振动方程

根据牛顿第二定律的转动形式:

$$ \tau = I \alpha $$

$$ I \frac{d^2\theta}{dt^2} + k_{\text{总}} \theta = 0 $$

$$ T = 2\pi \sqrt{\frac{I}{k_{\text{总}}}} $$

3. 转动惯量的测量

通过测量系统的振动周期$T$,可以计算出刚体的转动惯量:

$$ I = \frac{k_{\text{总}} T^2}{4\pi^2} $$

4. 扭转刚度的测量

细线的单根扭转刚度$k$可通过已知转动惯量的标准物体测量。将标准物体与三线摆系统相连,测量振动周期$T$,根据公式:

k = \frac{4\pi^2 I_{\text{标准}}}{3T_{\text{标准}}^2}

5. 三线摆的理论优势

平衡性高:三根等长细线均匀分布,使系统具有良好的对称性和稳定性。

振动纯粹性:由于刚体的重心位于三线构成的平面中心,系统主要以扭转振动为主,避免了多余的摆动。

误差小:相比单线摆法,三线摆对外界干扰(如空气阻力、晃动)的敏感性较低。


6. 实验误差分析

细线非理想弹性:细线的刚度可能不是常数,可能会随振幅变化。

周期测量误差:振动周期的测量精度会直接影响结果。

安装误差:细线的长度、分布不均会导致刚体不稳定。

空气阻力:空气阻力可能会影响振动的衰减和周期。


通过增加测量次数取平均值、减小振动幅度等措施,可以有效降低误差。

\subsubsection{落体法}

落体法的理论原理

落体法是一种通过自由下落过程中物体的运动特性来测量转动惯量的实验方法。该方法利用能量守恒定律或动力学方程,通过物体的下落加速度、角速度及相关几何参数计算转动惯量。

1. 实验装置

落体法通常由以下部分组成:

一个绕固定轴旋转的刚体(待测物体)。

一个通过细绳连接的配重块,细绳绕过刚体的圆柱部分。

一个测量配重块加速度或刚体角加速度的装置。


当配重块自由下落时,细绳对刚体产生一个力矩,刚体开始旋转,整个系统的运动遵循力学定律。

2. 理论基础

落体法的理论主要基于以下两个物理规律:

(1)牛顿第二定律

配重块的线性运动由牛顿第二定律描述:

$$ ma = mg - T $$

 $m$为配重块质量;

 $a$为配重块的加速度;

 $g$为重力加速度;

 $T$为细绳对配重块的拉力。


(2)刚体转动方程

刚体的转动由转动牛顿第二定律描述:

\tau = I \alpha

 $\tau=T\cdot R$是作用在刚体上的力矩;

 $I$是待测刚体的转动惯量;

 $\alpha$是刚体的角加速度;

 $R$是细绳绕刚体圆柱部分的半径。


此外,配重块的线性加速度$a$与刚体的角加速度$\alpha$满足几何关系:

$$ a = R \alpha $$

3. 转动惯量的计算

结合以上三个公式:

1. 配重块的动力学方程:

$$ I = R^2 \left( \frac{m g - m a}{a} \right) $$

$m$为配重块质量;

$g$为重力加速度;

$a$为配重块的线性加速度;

$R$为绳子绕刚体圆柱的半径。


4. 测量步骤

(1) 测量几何参数:精确测量绳子绕刚体圆柱的半径$R$和配重块的质量$m$。
(2) 测量加速度:通过传感器或视频分析记录配重块下落的加速度$a$。
(3) 计算转动惯量:代入公式计算刚体的转动惯量。
$$ I=R^2(\frac{mg-ma}{a}) $$


5. 优点与误差分析

优点

简单高效:实验装置简单,加速度的测量易于实现。

理论清晰:利用基本的牛顿运动定律和几何关系,公式推导简单明了。


误差来源

摩擦力影响:轴承摩擦和空气阻力可能导致测量误差。

加速度测量误差:加速度的测量精度直接影响计算结果。

绳子质量忽略:假设绳子质量远小于配重块质量,这可能带来系统误差。

非理想刚体:刚体可能存在形变或不均匀质量分布。

\section{现实意义}

总结

落体法通过结合配重块的线性运动与刚体的转动,利用简单的动力学方程精确测量转动惯量。该方法以其简洁性和高效性在物理实验中得到了广泛应用,同时对实验装置的摩擦等影响因素的校正是提高测量精度的重要环节。


\section{实验装置}
\subsection{硬件架构}


\subsection{软件架构}



\section{实验操作}
\subsection{扭摆法}
\subsubsection{实验方法}

\subsubsection{实验步骤}

\subsubsection{实验数据}

\subsubsection{实验结果}


\subsection{三线摆法}
\subsubsection{实验方法}

\subsubsection{实验步骤}

\subsubsection{实验数据}

\subsubsection{实验结果}


\subsection{落体法}
\subsubsection{实验方法}

\subsubsection{实验步骤}

\subsubsection{实验数据}

\subsubsection{实验结果}


\section{拓展:流体力学转动惯量探究}
\subsection{理论背景:液体在匀速旋转容器中转动惯量的计算}
\textbf{转动惯量的定义}:对于一个连续质量分布的系统,惯性矩 $I$ 可由以下积分公式给出:  
$$
I = \int_V r^2 \,dm
$$  
其中$r$是质量元 $dm$ 到旋转轴的距离,$V$ 是物体所占的体积,$dm = \rho \, dV$,其中 $\rho$ 是密度,$dV$ 是体积元。

\subsubsection{液体在旋转容器中的建模方法}

假设条件为:容器是一个理想的刚性结构,且绕特定轴旋转。液体的自由表面在旋转时形成稳定的抛物面形状。液体的密度 $\rho$ 是均匀的。

从《普通物理学实验1(H)》课程的“液体旋转实验”中,我们已经知道,液体自由表面的高度分布为: 
$$
h(r) = h_0 + \frac{\omega^2 r^2}{2g}
$$  

其中$h_0$是液体中心的高度,$\omega$ 是角速度,$g$ 是重力加速度,$r$ 是到旋转轴的径向距离。

\subsubsection{惯性矩的计算}

液体的惯性矩 $I$ 表示为:  
$$
I = \int_V r^2 \rho \, dV
$$  
其中:  
$$
dV = r \, dr \, d\theta \, dz
$$  
代入惯性矩公式:  
$$
I = \rho \int_0^{2\pi} \int_0^{R} \int_0^{h(r)} r^2 \cdot r \, dz \, dr \, d\theta
$$  
其中 $R$ 是容器的最大半径,$h(r)$ 是自由表面的高度。

\subsubsection{积分求解}
\begin{enumerate}
\item \textbf{对 $z$ 积分}:  
$$
\int_0^{h(r)} dz = h(r)
$$  

代入,得到:  
$$
I = \rho \int_0^{2\pi} \int_0^{R} r^3 h(r) \, dr \, d\theta
$$  

\item \textbf{将 $h(r)$ 代入}:  
$$
h(r) = h_0 + \frac{\omega^2 r^2}{2g}
$$  
因此:  
$$
I = \rho \int_0^{2\pi} \int_0^{R} r^3 \left( h_0 + \frac{\omega^2 r^2}{2g} \right) \, dr \, d\theta
$$  

\item \textbf{分解积分}:  
$$
I = \rho \int_0^{2\pi} d\theta \left[ \int_0^{R} r^3 h_0 \, dr + \int_0^{R} r^3 \frac{\omega^2 r^2}{2g} \, dr \right]
$$  

\item \textbf{计算径向积分}:  
对于 $\int_0^R r^3 \, dr$ 和 $\int_0^R r^5 \, dr$,我们分别得到:  
$$
\int_0^R r^3 \, dr = \frac{R^4}{4}, \quad \int_0^R r^5 \, dr = \frac{R^6}{6}
$$  
将结果代回:  
$$
I = \rho \cdot 2\pi \left[ h_0 \cdot \frac{R^4}{4} + \frac{\omega^2}{2g} \cdot \frac{R^6}{6} \right]
$$  

\item \textbf{化简最终结果}:  
$$
I = \frac{\pi \rho R^4 h_0}{2} + \frac{\pi \rho \omega^2 R^6}{6g}
$$  
得到我们最终理论推导的式子,即旋转液体转动惯量和转动容器半径、角速度的关系。
\end{enumerate}
\subsection{实验验证}
\subsubsection{实验方法}
根据理论建模中的条件,我们打印了相应的圆柱形容器,并在其中盛放液体:
\subsubsection{实验步骤}

\subsubsection{实验数据}

\subsubsection{实验结果}
\subsection{理论比较与误差分析}
\subsection{理论探究:匀加速旋转下的液体转动惯量}
\subsection{流体转动惯量理论的应用前景}



\section{分析与讨论}
\subsection{误差来源}

\subsection{误差分析}

\subsection{实验分析}


\section{实验总结}
\subsection{实验综述}

\subsection{不足之处}

\subsection{未来展望}

\section{课题总结}
\subsection{仪器系统照片}

\subsection{实验过程记录}

\subsection{感想}


\section{参考文献}

%%----------- 参考文献 -------------------%%
%在reference.bib文件中填写参考文献,此处自动生成
\reference

\end{document}