%!TeX program = xelatex
\documentclass[12pt,hyperref,a4paper,UTF8]{ctexart}
\usepackage{zjureport}
\usepackage{graphicx}
\usepackage{epstopdf}
\usepackage{subfloat}
\usepackage{booktabs}


%%-------------------------------正文开始---------------------------%%
\begin{document}

%%-----------------------封面--------------------%%
\cover

%%------------------摘要-------------%%
%\begin{abstract}
%
%在此填写摘要内容
%
%\end{abstract}

\thispagestyle{empty} % 首页不显示页码

%%--------------------------目录页------------------------%%
\newpage
\begin{spacing}{2}
\tableofcontents
\end{spacing}
\thispagestyle{empty}
\paragraph{}
\newpage
%%------------------------正文页从这里开始-------------------%


%%可选择这里也放一个标题
\begin{center}
    \title{ \Huge \textbf{{不规则物体的转动惯量测定方法}}}
\end{center}
\par
\begin{center}
    \textbf{Pan,He,Zhou,Zhou}
\end{center}
\paragraph{}

\section{背景知识}
\subsection{理论背景}

转动惯量又称为惯性矩,是物体的固有属性,描述了其在转动过程中对角速度变化的抵抗能力。在力学中,转动惯量通常用公式$I=\int r^2\text dm$表示,其中$r$是质量元$\text dm$到旋转轴的距离。它不仅取决于物体的质量分布,还与旋转轴的位置密切相关,因此轴的位置对转动惯量测量具有重要意义。

转动惯量的测量通常利用角动量守恒定律和牛顿第二定律的旋转形式。根据角动量定理,当合外力矩作用在刚体上时,刚体的角加速度$\alpha$与外力矩&\tau&之间满足$\tau=I\alpha$。通过已知外力矩和角加速度,可以反推刚体的转动惯量。

转动惯量的测定并非容易的问题,尤其是针对形状不规则的物体或拼接体。因此,采用扭摆法、三线摆法或落体法等间接方法在转动惯量的测定中备受欢迎。

\subsubsection{扭摆法}

扭摆法是一种利用扭转振动测量物体转动惯量的经典方法,其理论原理基于单摆运动与刚体转动的结合。扭摆的基本结构由悬挂刚体的细长弹性丝或扭转弹簧组成,当刚体被轻微旋转后,系统会产生周期性的扭转振动。

1. 扭转振动的基本原理

当刚体绕扭丝的轴发生微小转动时,弹性扭丝会产生恢复力矩$\tau$来抵抗这种转动,恢复力矩与角位移$\theta$成正比,即:

$$ \tau = -k \theta $$

根据刚体转动的牛顿第二定律:

$$ \tau = I \alpha $$

$$ I \frac{d^2\theta}{dt^2} + k\theta = 0 $$

该方程描述了刚体绕固定轴的简谐振动,其振动周期$T$为:

$$T = 2\pi \sqrt{\frac{I}{k}}$$

2. 测量扭转刚度 

扭转刚度$k$是弹性丝的固有属性,与其材料、直径、长度等参数有关。在实验中,可以通过已知转动惯量的标准物体测量扭转刚度:

$$k = \frac{4\pi^2 I_{\text{标准}}}{T_{\text{标准}}^2}$$

3. 测量未知物体的转动惯量

将待测物体安装在扭丝上,测量其扭转振动的周期 。通过已知的$T$值,利用下式计算转动惯量:

$$I = \frac{k T^2}{4\pi^2}$$

4. 复合物体的转动惯量

如果测量系统由扭丝、标准物体和待测物体组成,总转动惯量为:

$$I_{\text{总}} = I_{\text{标准}} + I_{\text{待测}}$$

5. 误差分析

使用扭摆法测量转动惯量时,主要误差来源包括:

扭丝非线性恢复力矩导致的系统误差。

振动周期测量中的读数误差。

扭丝刚度$k$计算误差。

系统外界扰动对振动周期的影响。


通过多次测量取平均值、减小扭丝非线性效应及减小振幅等措施可以有效降低误差。

总结

扭摆法通过测量振动周期并结合扭转刚度来精确计算物体的转动惯量。其优点在于结构简单、实验精度高,适用于多种刚体转动惯量的测量。

\subsubsection{三线摆法}

三线摆法是一种测量物体转动惯量的实验方法,其特点是结构简单、测量精度高,广泛应用于实验力学中。三线摆通过悬挂系统产生扭转振动,通过测量振动周期,结合系统的几何和物理参数计算待测物体的转动惯量。

1. 三线摆的基本结构

三线摆由三个等长且均匀分布的细线(或金属丝)悬挂一个刚体构成。细线的另一端固定在支架上。系统绕通过刚体重心且与三线平面垂直的轴进行自由扭转振动。刚体的重心与三线摆系统的对称性和力学平衡性保证了振动的稳定性。

2. 振动原理

三线摆的运动是绕垂直于线摆平面轴的扭转振动,其振动由刚体的转动惯量$I$与细线的弹性扭转力共同决定。当刚体偏离平衡位置发生转动时,细线的扭转恢复力矩使其产生简谐振动。

恢复力矩

细线因扭转产生的恢复力矩$\tau$与角位移$\theta$成正比:

$$ \tau = -k_{\text{总}} \theta $$

$$ k_{\text{总}} = 3k $$

振动方程

根据牛顿第二定律的转动形式:

$$ \tau = I \alpha $$

$$ I \frac{d^2\theta}{dt^2} + k_{\text{总}} \theta = 0 $$

$$ T = 2\pi \sqrt{\frac{I}{k_{\text{总}}}} $$

3. 转动惯量的测量

通过测量系统的振动周期$T$,可以计算出刚体的转动惯量:

$$ I = \frac{k_{\text{总}} T^2}{4\pi^2} $$

4. 扭转刚度的测量

细线的单根扭转刚度$k$可通过已知转动惯量的标准物体测量。将标准物体与三线摆系统相连,测量振动周期$T$,根据公式:

k = \frac{4\pi^2 I_{\text{标准}}}{3T_{\text{标准}}^2}

5. 三线摆的理论优势

平衡性高:三根等长细线均匀分布,使系统具有良好的对称性和稳定性。

振动纯粹性:由于刚体的重心位于三线构成的平面中心,系统主要以扭转振动为主,避免了多余的摆动。

误差小:相比单线摆法,三线摆对外界干扰(如空气阻力、晃动)的敏感性较低。


6. 实验误差分析

细线非理想弹性:细线的刚度可能不是常数,可能会随振幅变化。

周期测量误差:振动周期的测量精度会直接影响结果。

安装误差:细线的长度、分布不均会导致刚体不稳定。

空气阻力:空气阻力可能会影响振动的衰减和周期。


通过增加测量次数取平均值、减小振动幅度等措施,可以有效降低误差。

\subsubsection{落体法}

落体法的理论原理

落体法是一种通过自由下落过程中物体的运动特性来测量转动惯量的实验方法。该方法利用能量守恒定律或动力学方程,通过物体的下落加速度、角速度及相关几何参数计算转动惯量。

1. 实验装置

落体法通常由以下部分组成:

一个绕固定轴旋转的刚体(待测物体)。

一个通过细绳连接的配重块,细绳绕过刚体的圆柱部分。

一个测量配重块加速度或刚体角加速度的装置。


当配重块自由下落时,细绳对刚体产生一个力矩,刚体开始旋转,整个系统的运动遵循力学定律。

2. 理论基础

落体法的理论主要基于以下两个物理规律:

(1)牛顿第二定律

配重块的线性运动由牛顿第二定律描述:

$$ ma = mg - T $$

 $m$为配重块质量;

 $a$为配重块的加速度;

 $g$为重力加速度;

 $T$为细绳对配重块的拉力。


(2)刚体转动方程

刚体的转动由转动牛顿第二定律描述:

$$  \tau = I \alpha $$

 $\tau=T\cdot R$是作用在刚体上的力矩;

 $I$是待测刚体的转动惯量;

 $\alpha$是刚体的角加速度;

 $R$是细绳绕刚体圆柱部分的半径。


此外,配重块的线性加速度$a$与刚体的角加速度$\alpha$满足几何关系:

$$ a = R \alpha $$

3. 转动惯量的计算

结合以上三个公式:

1. 配重块的动力学方程:

$$ I = R^2 \left( \frac{m g - m a}{a} \right) $$

$m$为配重块质量;

$g$为重力加速度;

$a$为配重块的线性加速度;

$R$为绳子绕刚体圆柱的半径。


4. 测量步骤

(1) 测量几何参数:精确测量绳子绕刚体圆柱的半径$R$和配重块的质量$m$。
(2) 测量加速度:通过传感器或视频分析记录配重块下落的加速度$a$。
(3) 计算转动惯量:代入公式计算刚体的转动惯量。
$$ I=R^2(\frac{mg-ma}{a}) $$


5. 优点与误差分析

优点

简单高效:实验装置简单,加速度的测量易于实现。

理论清晰:利用基本的牛顿运动定律和几何关系,公式推导简单明了。


误差来源

摩擦力影响:轴承摩擦和空气阻力可能导致测量误差。

加速度测量误差:加速度的测量精度直接影响计算结果。

绳子质量忽略:假设绳子质量远小于配重块质量,这可能带来系统误差。

非理想刚体:刚体可能存在形变或不均匀质量分布。


总结

落体法通过结合配重块的线性运动与刚体的转动,利用简单的动力学方程精确测量转动惯量。该方法以其简洁性和高效性在物理实验中得到了广泛应用,同时对实验装置的摩擦等影响因素的校正是提高测量精度的重要环节。


\section{实验装置}
\subsection{硬件架构}
本实验中主要的器材包括扭摆法器材、三线摆法器材和落体法器材。

此外,我们还通过Blender建模,3D打印制造了质量与体积已知的标准元件。用于转动惯量的测定实验。

\subsection{软件架构}
采用Blender进行模拟,研究不规则物体与流体的转动性质。


\section{实验操作}
\subsection{扭摆法}
\subsubsection{实验方法}

\subsubsection{实验步骤}

\subsubsection{实验数据}

\subsubsection{实验结果}

\subsubsection{误差来源与分析}

\subsection{三线摆法}
\subsubsection{实验方法}

\subsubsection{实验步骤}

\subsubsection{实验数据}

\subsubsection{实验结果}

\subsubsection{误差来源与分析}


\subsection{落体法}
\subsubsection{实验方法}
\begin{itemize}
    \item \textbf{加速过程}:在外力(如砝码拉力)和摩擦力矩的共同作用下,系统产生角加速度 $\alpha_1$。
    \item \textbf{减速过程}:祛码脱落后,系统仅在摩擦力矩作用下继续转动,产生角加速度 $\alpha_2$。
\end{itemize}

系统的转动惯量 $I$ 通过下式计算:
\begin{equation}
    I = \frac{m r^2 (\alpha_1 + \alpha_2)}{\alpha_1 \alpha_2},
\end{equation}
其中:
\begin{itemize}
    \item $m$:砝码质量;
    \item $r$:塔轮半径;
    \item $\alpha_1$:加速过程的角加速度;
    \item $\alpha_2$:减速过程的角加速度。
\end{itemize}

\subsubsection{实验步骤}
\begin{enumerate}
    \item \textbf{实验准备}
    \begin{itemize}
        \item 调整实验装置,确保水平稳定,减少外部干扰。
        \item 安装待测刚体圆盘与祛码,测量祛码的质量 $m$ 和塔轮的半径 $r$。
    \end{itemize}

    \item \textbf{加速过程测量}
    \begin{itemize}
        \item 挂上砝码,释放系统,系统开始加速转动。
        \item 测量系统完成一定圈数所需的时间 $t_1$。
        \item 计算加速角加速度 $\alpha_1$,公式如下:
        \begin{equation}
            \alpha_1 = \frac{\theta}{t_1^2},
        \end{equation}
        其中 $\theta$ 为角位移。
    \end{itemize}

    \item \textbf{减速过程测量}
    \begin{itemize}
        \item 祛码脱落后,系统在摩擦力矩作用下继续转动。
        \item 测量系统完成相同圈数所需的时间 $t_2$。
        \item 计算减速角加速度 $\alpha_2$,公式如下:
        \begin{equation}
            \alpha_2 = \frac{\theta}{t_2^2}.
        \end{equation}
    \end{itemize}

    \item \textbf{数据处理}
    \begin{itemize}
        \item 将测得的 $\alpha_1$ 和 $\alpha_2$ 代入转动惯量计算公式:
        \begin{equation}
            I = \frac{m r^2 (\alpha_1 + \alpha_2)}{\alpha_1 \alpha_2}.
        \end{equation}
        \item 重复实验多次,取平均值以减少实验误差。
    \end{itemize}
    \end{itemize}
\end{enumerate}

\subsubsection{实验数据}
    \begin{itemize}
        \item \textbf{牵引力不同,半径相同}
            \item $m=84.90\text g$
            
                \begin{table}[h!]
                \centering
                \begin{tabular}{cccccc}
                \toprule
                \textbf{周期(转)} & \textbf{1} & \textbf{2} & \textbf{3} & \textbf{4} & \textbf{5} \\
                \midrule
                1  & 1.696 & 1.182 & 1.553 & 1.191 \\
                2  & 1.038 & 0.898 & 1.078 & 0.922 \\
                3  & 0.829 & 0.762 & 0.944 & 0.762 \\
                4  & 0.692 & 0.638 & 0.775 & 0.681 \\
                5  & 0.630 & 0.597 & 0.682 & 0.604 \\
                6  & 0.561 & 0.547 & 0.592 & 0.551 \\
                7  & 0.497 & 0.490 & 0.550 & 0.460 \\
                8  & 0.457 & 0.476 & 0.528 & 0.523 \\
                9  & 0.442 & 0.451 & 0.495 & 0.438 \\
                10 & 0.423 & 0.427 & 0.464 & 0.439 \\
                11 & 0.398 & 0.405 & 0.437 & 0.419 \\
                12 & 0.385 & 0.379 & 0.421 & 0.402 \\
                13 & 0.369 & 0.377 & 0.406 & 0.385 \\
                14 & 0.330 & 0.364 & 0.379 & 0.369 \\
                15 & 0.319 & 0.347 & 0.372 & 0.342 \\
                16 & 0.324 & 0.320 & 0.330 & 0.323 \\
                17 & 0.320 & 0.323 & 0.323 & 0.323 \\
                18 & 0.320 & 0.319 & 0.319 & 0.319 \\
                19 & 0.311 & 0.315 & 0.320 & 0.317 \\
                20 & 0.310 & 0.306 & 0.323 & 0.311 \\
                \midrule
                \textbf{平均值} & 0.53255 & 0.49615 & 0.56455 & 0.504 \\
                \bottomrule
                \end{tabular}
                \caption{实验周期数据表2}
                \end{table}

            \item $m=59.86\text g$
                \begin{table}[h!]
                    \centering
                    \begin{tabular}{cccccc}
                    \toprule
                    \textbf{周期(转)} & \textbf{1} & \textbf{2} & \textbf{3} & \textbf{4} & \textbf{5} \\
                    \midrule
                    1  & 1.637 & 1.393 & 1.720 & 2.455 \\
                    2  & 1.134 & 1.056 & 1.197 & 1.315 \\
                    3  & 0.927 & 0.858 & 1.085 & 0.937 \\
                    4  & 0.801 & 0.753 & 0.908 & 0.905 \\
                    5  & 0.705 & 0.701 & 0.805 & 0.765 \\
                    6  & 0.667 & 0.623 & 0.745 & 0.700 \\
                    7  & 0.615 & 0.599 & 0.679 & 0.630 \\
                    8  & 0.558 & 0.560 & 0.552 & 0.603 \\
                    9  & 0.544 & 0.522 & 0.621 & 0.565 \\
                    10 & 0.517 & 0.501 & 0.557 & 0.510 \\
                    11 & 0.495 & 0.485 & 0.513 & 0.507 \\
                    12 & 0.462 & 0.465 & 0.505 & 0.488 \\
                    13 & 0.443 & 0.448 & 0.484 & 0.464 \\
                    14 & 0.440 & 0.431 & 0.466 & 0.436 \\
                    15 & 0.426 & 0.404 & 0.448 & 0.435 \\
                    16 & 0.412 & 0.385 & 0.434 & 0.404 \\
                    17 & 0.392 & 0.387 & 0.404 & 0.400 \\
                    18 & 0.385 & 0.385 & 0.386 & 0.393 \\
                    19 & 0.375 & 0.372 & 0.385 & 0.379 \\
                    20 & 0.369 & 0.341 & 0.386 & 0.373 \\
                    \midrule
                    \textbf{平均值} & 0.6152 & 0.58345 & 0.664 & 0.68285 \\
                    \bottomrule
                    \end{tabular}
                    \caption{实验周期数据表3}
                    \end{table}

            \item $m=59.86\text g$
                \begin{table}[h!]
                    \centering
                    \begin{tabular}{ccccc}
                    \toprule
                    \textbf{周期(转)} & \textbf{1} & \textbf{2} & \textbf{3} & \textbf{4} \\
                    \midrule
                    1  & 2.718 & 4.065 & 2.753 \\
                    2  & 2.084 & 2.411 & 2.013 \\
                    3  & 1.701 & 1.888 & 1.652 \\
                    4  & 1.481 & 1.622 & 1.440 \\
                    5  & 1.339 & 1.446 & 1.321 \\
                    6  & 1.152 & 1.315 & 1.217 \\
                    7  & 1.177 & 1.209 & 1.115 \\
                    8  & 1.057 & 1.128 & 1.053 \\
                    9  & 1.006 & 1.066 & 1.007 \\
                    10 & 0.943 & 1.006 & 0.882 \\
                    11 & 0.926 & 0.949 & 0.953 \\
                    12 & 0.891 & 0.924 & 0.882 \\
                    13 & 0.862 & 0.889 & 0.834 \\
                    14 & 0.763 & 0.855 & 0.817 \\
                    15 & 0.836 & 0.825 & 0.787 \\
                    16 & 0.772 & 0.782 & 0.767 \\
                    17 & 0.758 & 0.771 & 0.746 \\
                    18 & 0.725 & 0.751 & 0.725 \\
                    19 & 0.712 & 0.734 & 0.705 \\
                    20 & 0.701 & 0.711 & 0.664 \\
                    \midrule
                    \textbf{总和} & 22.604 & 25.347 & 22.335 \\
                    \bottomrule
                    \end{tabular}
                    \caption{实验周期数据表}
                    \end{table}
                    

        \item \textbf{半径相同,牵引力不同}
            \item $r=$
                \begin{table}[h!]
                    \centering
                    \begin{tabular}{ccccc}
                    \toprule
                    \textbf{周期(转)/s} & \textbf{1} & \textbf{2} & \textbf{3} & \textbf{4} \\
                    \midrule
                    1  & 1.193 & 4.194 & 1.433 & 1.280 \\
                    2  & 0.887 & 7.940 & 0.895 & 0.862 \\
                    3  & 0.692 & 2.215 & 0.726 & 0.700 \\
                    4  & 0.577 & 0.980 & 0.639 & 0.580 \\
                    5  & 0.534 & 0.760 & 0.554 & 0.531 \\
                    6  & 0.484 & 0.645 & 0.511 & 0.484 \\
                    7  & 0.428 & 0.576 & 0.454 & 0.425 \\
                    8  & 0.423 & 0.520 & 0.429 & 0.425 \\
                    9  & 0.387 & 0.447 & 0.413 & 0.382 \\
                    10 & 0.375 & 0.434 & 0.390 & 0.380 \\
                    11 & 0.363 & 0.418 & 0.372 & 0.358 \\
                    12 & 0.332 & 0.390 & 0.335 & 0.338 \\
                    13 & 0.320 & 0.377 & 0.316 & 0.315 \\
                    14 & 0.322 & 0.360 & 0.322 & 0.323 \\
                    15 & 0.301 & 0.339 & 0.316 & 0.321 \\
                    16 & 0.265 & 0.317 & 0.275 & 0.293 \\
                    17 & 0.327 & 0.317 & 0.275 & 0.293 \\
                    \bottomrule
                    \end{tabular}
                    \caption{实验周期数据表4}
                    \end{table}
                    
            \item $r=$
            \begin{table}[h!]
                \centering
                \begin{tabular}{cccc}
                \toprule
                \textbf{周期(转)/s} & \textbf{1} & \textbf{2} & \textbf{3} \\
                \midrule
                1  & 0.949 & 1.885 & 0.957 \\
                2  & 0.642 & 0.878 & 0.708 \\
                3  & 0.621 & 0.664 & 0.586 \\
                4  & 0.515 & 0.559 & 0.484 \\
                5  & 0.464 & 0.501 & 0.464 \\
                6  & 0.426 & 0.443 & 0.426 \\
                7  & 0.393 & 0.419 & 0.381 \\
                8  & 0.373 & 0.378 & 0.373 \\
                9  & 0.349 & 0.368 & 0.348 \\
                10 & 0.335 & 0.318 & 0.336 \\
                11 & 0.320 & 0.315 & 0.313 \\
                12 & 0.306 & 0.312 & 0.305 \\
                13 & 0.272 & 0.302 & 1.501 \\
                \bottomrule
                \end{tabular}
                \caption{实验周期数据表5}
                \end{table}
                
    \end{itemize}

\subsubsection{实验结果}
    \begin{itemize}
        \item \textbf{牵引力不同,半径相同}
            \begin{table}[h!]
                \centering
                \begin{tabular}{cccccc}
                \toprule
                \textbf{砝码质量/g} & \textbf{角加速度/$(\text{s}^{-2})$} & \textbf{转动惯量测定值/$(\text{kg}\cdot \text{m}^2)$} & \textbf{转动惯量理论值/$(\text{kg}\cdot \text{m}^2)$} & \textbf{相对误差} \\
                \midrule
                84.90 & 17.85523372 & 0.003977143 & 0.003938045 & 0.99\% \\
                59.86 & 12.17850629 & 0.003881379 & 0.003938045 & -1.44\% \\
                19.83 & 3.437479696 & /          & /          & / \\
                \bottomrule
                \end{tabular}
                \caption{转动惯量测量实验数据}
                \end{table}
                
        \item \textbf{半径相同,牵引力不同}
            \begin{table}[h!]
                \centering
                \begin{tabular}{cccccc}
                \toprule
                \textbf{旋转半径/mm} & \textbf{角加速度/$(\text{s}^{-2})$} & \textbf{转动惯量测定值/$(\text{kg}\cdot \text{m}^2)$} & \textbf{转动惯量理论值/$(\text{kg}\cdot \text{m}^2)$} & \textbf{相对误差} \\
                \midrule
                15.37 & 17.85523372 & 0.003977143 & 0.003938045 & 0.99\% \\
                20.10 & 19.53034708 & 0.003970183 & 0.003938045 & 0.82\% \\
                25.61 & 23.80698023 & 0.003816255 & 0.003938045 & -3.09\% \\
                \bottomrule
                \end{tabular}
                \caption{不同旋转半径下的转动惯量测量实验数据}
                \end{table}
                
    \end{itemize}


\subsubsection{误差来源与分析}
    \begin{itemize}
        \item \textbf{主要误差来源}
        \begin{itemize}
            \item 启动阶段的摩擦力矩变化;
            \item 测量时间误差。
        \end{itemize}
        \item \textbf{误差控制}
        \begin{itemize}
            \item 保持加速和减速过程的转速接近;
            \item 使用高精度计时器记录时间。
        \end{itemize}

\section{拓展:流体力学转动惯量探究}
        \subsection{理论背景:液体在匀速旋转容器中转动惯量的计算}
        \textbf{转动惯量的定义}:对于一个连续质量分布的系统,惯性矩 $I$ 可由以下积分公式给出:  
        $$
        I = \int_V r^2 \,dm
        $$  
        其中$r$是质量元 $dm$ 到旋转轴的距离,$V$ 是物体所占的体积,$dm = \rho \, dV$,其中 $\rho$ 是密度,$dV$ 是体积元。
        
        \subsubsection{液体在旋转容器中的建模方法}
        
        假设条件为:容器是一个理想的刚性结构,且绕特定轴旋转。液体的自由表面在旋转时形成稳定的抛物面形状。液体的密度 $\rho$ 是均匀的。
        
        从《普通物理学实验1(H)》课程的“液体旋转实验”中,我们已经知道,液体自由表面的高度分布为: 
        $$
        h(r) = h_0 + \frac{\omega^2 r^2}{2g}
        $$  
        
        其中$h_0$是液体中心的高度,$\omega$ 是角速度,$g$ 是重力加速度,$r$ 是到旋转轴的径向距离。
        
        \subsubsection{惯性矩的计算}
        
        液体的惯性矩 $I$ 表示为:  
        $$
        I = \int_V r^2 \rho \, dV
        $$  
        其中:  
        $$
        dV = r \, dr \, d\theta \, dz
        $$  
        代入惯性矩公式:  
        $$
        I = \rho \int_0^{2\pi} \int_0^{R} \int_0^{h(r)} r^2 \cdot r \, dz \, dr \, d\theta
        $$  
        其中 $R$ 是容器的最大半径,$h(r)$ 是自由表面的高度。
        
        \subsubsection{积分求解}
        \begin{enumerate}
        \item \textbf{对 $z$ 积分}:  
        $$
        \int_0^{h(r)} dz = h(r)
        $$  
        
        代入,得到:  
        $$
        I = \rho \int_0^{2\pi} \int_0^{R} r^3 h(r) \, dr \, d\theta
        $$  
        
        \item \textbf{将 $h(r)$ 代入}:  
        $$
        h(r) = h_0 + \frac{\omega^2 r^2}{2g}
        $$  
        因此:  
        $$
        I = \rho \int_0^{2\pi} \int_0^{R} r^3 \left( h_0 + \frac{\omega^2 r^2}{2g} \right) \, dr \, d\theta
        $$  
        
        \item \textbf{分解积分}:  
        $$
        I = \rho \int_0^{2\pi} d\theta \left[ \int_0^{R} r^3 h_0 \, dr + \int_0^{R} r^3 \frac{\omega^2 r^2}{2g} \, dr \right]
        $$  
        
        \item \textbf{计算径向积分}:  
        对于 $\int_0^R r^3 \, dr$ 和 $\int_0^R r^5 \, dr$,我们分别得到:  
        $$
        \int_0^R r^3 \, dr = \frac{R^4}{4}, \quad \int_0^R r^5 \, dr = \frac{R^6}{6}
        $$  
        将结果代回:  
        $$
        I = \rho \cdot 2\pi \left[ h_0 \cdot \frac{R^4}{4} + \frac{\omega^2}{2g} \cdot \frac{R^6}{6} \right]
        $$  
        
        \item \textbf{化简最终结果}:  
        $$
        I = \frac{\pi \rho R^4 h_0}{2} + \frac{\pi \rho \omega^2 R^6}{6g}
        $$  
        得到我们最终理论推导的式子,即旋转液体转动惯量和转动容器半径、角速度的关系。
        \end{enumerate}

        \subsection{理论比较与误差分析}
        Dodge [2]、Ibrahim [5]、Faltinsen 与 Timokha [6] 对液体阻尼的相关问题进行了全面而系统的阐述,  
        归纳容器内液体的阻尼来自于四个方面:
        1. 液体与容器固壁之间的边界层摩擦阻尼;
        2. 液体内涡的黏性阻尼;
        3. 自由表面阻尼;
        4. 其他三维阻尼。
        
        一般认为液体与容器固壁之间的边界层阻尼是主要作用。  
        对于小幅液体晃动而言,可采用无阻尼不可压缩的势流理论来研究液体晃动,采用速度势的数学模型来描述液体运动,从而可以得到液体的运动方程:
        
        \begin{equation}
        \dot{q_j} + \omega_j^2 q_j = f_j(t), \quad (j = 1, 2, 3, \dots)
        \end{equation}
        
        式中,$q_j$ 为液体系统的第 $j$ 阶自然频率,$\omega_j$ 为第 $j$ 阶固有频率,$f_j(t)$ 为第 $j$ 阶外力。  
        方程(1) 表示了一个无阻尼液体系统的强迫振动方程。当需要考虑液体的晃动阻尼时,可在方程(1)上增加一个阻尼项 $2\zeta_j\omega_j \dot{q_j}$,即有:
        
        \begin{equation}
        \dot{q_j} + 2\zeta_j\omega_j\dot{q_j} + \omega_j^2 q_j = f_j(t), \quad (j = 1, 2, 3, \dots)
        \end{equation}
        
        式中,$\zeta_j$ 为第 $j$ 阶模态的阻尼比系数,主要由阻尼的晃动可以由公式(3)来确定。  
        这种液体晃动阻尼可以通过边界层理论的边界层阻尼来分析。基于 Stokes 边界层理论,可得到容器的边界层阻尼的估计公式为:
        
        \begin{equation}
        \zeta_b = \frac{1}{2} \sqrt{\frac{\nu}{2\omega}} \frac{\int_{\partial S} |\nabla \varphi|^2 ds}{\int_{\int V} |\nabla \varphi|^2 dv}
        \end{equation}
        
        式中:
        - $V$ 为液体空间区域;
        - $\partial S$ 为容器的边界;
        - $\varphi$ 表示了速度势的幅值函数;
        - $\nu$ 为液体运动黏度;
        - $\omega$ 为液体晃动圆频率。
        
        液体内涡的黏性阻尼比可以采用下列黏滞参数计算[6]:
        
        \begin{equation}
        \zeta_v = \frac{\nu}{\omega} \frac{\int_{\int V} \left[ 
        \left(\frac{\partial^2 \varphi}{\partial x^2}\right)^2 + 
        \left(\frac{\partial^2 \varphi}{\partial y^2}\right)^2 + 
        \left(\frac{\partial^2 \varphi}{\partial z^2}\right)^2 + 
        2\left(\frac{\partial^2 \varphi}{\partial x \partial y}\right)^2 + 
        2\left(\frac{\partial^2 \varphi}{\partial y \partial z}\right)^2 + 
        2\left(\frac{\partial^2 \varphi}{\partial z \partial x}\right)^2
        \right] dv}
        {\int_{\int V} |\nabla \varphi|^2 dv}
        \end{equation}
        
        一般认为,以上液体内部的体阻尼比相较于边界层阻尼要小得多,体阻尼可忽略不计。  
        对于短矩形容器,根据公式(3)(4)可得到阻尼比的解析公式。对于非矩形的容器,可采用数值方法计算。

\section{分析与讨论}

\subsection{理论补充}
\subsubsection{基于 LaGrange 方程的建模方法}
假设非线性的转动惯量测定系统具有$n$个广义坐标系,那么其 LaGrange 方程为:
$$ \frac{\text d}{\text dt}(\frac{\partial E}{\partial\overset{.}q_i})-\frac{\partial E}{\partial q_i}+\frac{\partial U}{\partial q_i}=Q_i $$
其中$n$为系统的自由度数目,$q_i$为第$i$个广义坐标,$E$为系统的动能,$U$为系统的势能,$Q_i$为与广义坐标$q_i$对应的广义力,包括阻尼和外加激振力。
取扭摆角$\theta$为广义坐标,代入 LaGrange 方程可得:
$$ \frac{\text d}{\text dt}(\frac{\partial E}{\partial\overset{.}\theta})-\frac{\partial E}{\partial \theta}+\frac{\partial U}{\partial \theta}=Q $$

若系统中的非线性因素是弱的,则微分方程可以表示为:
$$ \overset{..}\theta+\omega_0^2\theta=\varepsilon f(\theta,\overset{.}\theta) $$
其中,$\varepsilon$是小参数,$f(\theta,\overset{.}\theta)$是$\theta,\overset{.}\theta$的非线性解析函数。

该方程的解可以表示为:
$$ \theta=a(t)\cos[\omega_0t+\phi(t)] $$
即把振幅和相位看成时间$t$的函数。该方法称为慢变参数法。
\subsubsection{Runge-Kutta 法}
\subsubsection*{Runge-Kutta法简要介绍}

Runge-Kutta法(简称RK法)是一类用于数值求解常微分方程初值问题的重要算法。它通过引入多步逼近技术,在提高数值解精度的同时保持较高的计算效率,被广泛应用于科学与工程计算中,对于上文提及的 LaGrange 建模方法具有借鉴意义。

\paragraph{问题形式}
Runge-Kutta法主要用于求解以下形式的初值问题:
\[
\frac{dy}{dx} = f(x, y), \quad y(x_0) = y_0,
\]
其中,\( f(x, y) \) 是已知函数,\( y(x) \) 是待求解的函数。

\paragraph{基本思想}
RK方法通过在每一步内结合多个点的斜率(即 \( f(x, y) \) 的值)来计算步进,从而提高解的精度。最简单的RK方法(如欧拉法)只利用当前点的斜率,而高阶RK方法(如经典的四阶Runge-Kutta方法)会结合多个中间点的信息。

\paragraph{经典的四阶Runge-Kutta方法}
四阶Runge-Kutta法是最常用的RK方法之一,其公式为:
\[
k_1 = h f(x_n, y_n),
\]
\[
k_2 = h f\left(x_n + \frac{h}{2}, y_n + \frac{k_1}{2}\right),
\]
\[
k_3 = h f\left(x_n + \frac{h}{2}, y_n + \frac{k_2}{2}\right),
\]
\[
k_4 = h f(x_n + h, y_n + k_3),
\]
\[
y_{n+1} = y_n + \frac{1}{6}(k_1 + 2k_2 + 2k_3 + k_4),
\]
其中:
\begin{itemize}
    \item \( h \) 是步长;
    \item \( k_1, k_2, k_3, k_4 \) 是中间斜率的加权项。
\end{itemize}

该方法利用当前点和三个中间点的斜率估算出下一点的值,具有四阶精度(即局部截断误差为 \( O(h^5) \),全局误差为 \( O(h^4) \))。

\paragraph{总结}
Runge-Kutta法是求解常微分方程的一种高效且精确的方法,其四阶形式因计算精度与效率的良好平衡,在实际应用中极为常见。


\subsection{实验分析}


\subsection{讨论优化}
\subsubsection{实验样件的径向偏移}
根据平行轴定理,径向偏移引起的转动惯量测量误差可以表示为:
$$ \delta=\frac{\pm 2m_AL\Delta L+m_A\Delta L^2}{J_A+m_AL^2} $$

其中$m_A$为实验样件A的质量,$J_A$为实验样件A关于自身中心线的转动惯量。如果$\Delta L<<L$,那么误差公式可以简化为:
$$ -2\frac{\Delta L}{L}<\delta<2\frac{\Delta L}{L} $$
\section{实验总结}
\subsection{实验综述}
在本实验中,小组成员通过探索扭摆法、三线摆法和落体法三种基础的转动惯量测定方法。比较了转动惯量测定方法各自的长处。同时,小组成员利用Blender平台模拟了刚体的转动过程,为今后转动惯量的测定实验提供了一定的参考和借鉴;此外,还利用理论推导与三维建模打印的方法,研究了流体等非刚体旋转时的性质,具有理论探索上的创新性与突破性。
\subsection{不足之处}
    \begin{itemize}
        \item \textbf{三线摆法}
        在三线摆实验部分,实际上还存在许多的改进空间,如探究摆角对于实验结果的影响,在何种摆角范围内会得到最佳的测量条件;另外,实验并没有探究悬盘偏心平动对于实验结果准确度的影响。
        此外,实验也没有对不规则物体的转动惯量进行测量,理论来讲,可以通过悬挂法或者其他方法确定不规则物体的重心,放置于下悬盘的中心轴处就可以测量出该不规则物体的转动惯量,但是由于不规则物体的转动惯量的理论值难以计算,所以最终难以判断测量值与理论值的误差,因此并没有设计这一部分的实验。
    \end{itemize}
\subsection{未来展望}
    受限于实验仪器和实验时间不足,本次实验仍有许多可扩展的空间,期望之后可以引入高精度的计时技术,同时优化摆线材料与测量仪器,并改进测量技术,使实验结果更符合预期。通过不断的技术创新和实验优化,三线摆法将为转动惯量测量领域带来更多的突破与进步。相信在未来,三线摆法不仅将在科研和工程领域中发挥更大的作用,还将在物理教育和跨学科研究中展现出新的潜力。
\section{课题总结}
\subsection{仪器系统照片}

\subsection{实验过程记录}

\subsection{感想}
\begin{itemize}
    \item \textbf{何德怀}
    通过这次为期七周的实验,对于转动惯量的研究方法有了更深刻的认识,此外,对于大型实验的实验设计、进度开展也有了更明确的规划。

\end{itemize}

\section{参考文献}

%%----------- 参考文献 -------------------%%
%在reference.bib文件中填写参考文献,此处自动生成
\reference

\end{document}
